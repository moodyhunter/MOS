\chapter{Lab 2 - Processes}

Congratulations on completing the first lab! Hopefully you have had a chance to play
around with the MOS kernel and have a better understanding of its structure.

In this lab, we will be looking at the process management in MOS.

\section{Files in MOS that relate to process management}

\begin{figure}[ht]
    \centering
    \raisebox{-.5\height}{\includegraphics[width=0.4\textwidth]{assets/c2.headers.png}}
    \raisebox{-.5\height}{\includegraphics[width=0.4\textwidth]{assets/c2.sources.png}}
    \caption{The header and source files in \texttt{kernel/tasks}}
    \label{fig:mos-process-management-files}
\end{figure}

The process management in MOS is implemented in the \texttt{kernel/tasks} directory, and the
corresponding header files are in \texttt{kernel/include/private/mos/tasks} as shown in
Figure \ref{fig:mos-process-management-files}.

\begin{note}
    \item The \texttt{kernel/include/private} directory is a convension used in MOS, which
    contains the header files that are \textbf{not} meant to be used by userspace programs.
\end{note}

\section{Process Control Blocks}

You have learnt about this in the lecture: `a process is a program in execution', and a process
control block (PCB) is a data structure that contains information about a process.

Now, let's look at the PCB structure in MOS. The PCB is defined as a struct named \texttt{process\_t}
in \texttt{task\_types.h}.

Several fields are defined in PCB, but we will only look at the ones that are relevant to
process management, which are:

\begin{itemize}
    \item \texttt{pid}: The process id.

          Each process has a unique id, this is what MOS uses to identify a process.\\
          In MOS, the process id is an unsigned 32-bit integer, and this number always
          increases.

    \item \texttt{name}: The name of the process, nothing interesting here.

    \item \texttt{parent}: The parent of the process, this is a pointer to the PCB of the parent.

          Unlike most animals, a process can only have one parent. The \texttt{init}
          process lives at the root of the process tree, it's fun to think about it being its own
          parent.

    \item \texttt{files}: A list of opened files.

          This is a statically allocated array of size \texttt{MOS\_PROCESS\_MAX\_OPEN\_FILES},
          the value of which is determined by the kernel configuration.

          Each time a file is opened, an \texttt{io\_t} that represents an IO object (file, pipe, etc.)
          is added to this array, and the index of such an object is returned to the user program.

          This index is called a `file descriptor', or `fd' for short. The user program can use
          this number to refer to the file in the future.

          The \texttt{io\_t} structure is defined in \texttt{kernel/include/private/mos/io/io.h},
          the details of which is probably beyond the scope of this lab.

    \item \texttt{threads}: A list of threads.

          To be honest, this list is unnecessary, because we already have a thread table elsewhere
          in MOS, the list here is just a convenience for the kernel to iterate over all threads
          of a specific process, (think about it as a cache).

    \item \texttt{pagetable}, \texttt{mmaps}: The page table and a list of mapped memory regions,
          these two will be discussed in later chapters.
\end{itemize}

\section{Thread Control Blocks}

You may have noticed that the PCB above is \textit{little bit} different from the one in the
lecture: It has neither `process state' nor `stack/heap/text memory regions'.

Because, in MOS, a thread is basic unit of execution instead of a process. Thus it's the thread that has a
`state' and a `stack', and the process is just a collection of threads.

The thread control block is defined in the same file as the PCB. You can see that it contains
a \texttt{tid} for a thread id, and a \texttt{state} field to indicate the state of the thread,
its owner process, two stacks for the kernel and user mode respectively. There's also a
\texttt{context} pointer, this will be used in context switching, we'll look at that later.

\section{Process Creation}

To create a process, the following steps are taken:

\begin{enumerate}
    \item Load the program into the memory.
    \item Create a PCB and populate the fields.
          \begin{enumerate}
              \item Allocate and initialize the PCB.
              \item Allocate a user-mode page table.
              \item Parse the program and map corresponding memory regions into the page table.
              \item Allocate a heap for the process in the user-mode page table.
                    (part 2 in Figure \ref{fig:mos-process-memory-layout})
              \item Add the PCB to the process table.
          \end{enumerate}
    \item Create a main thread.
          \begin{enumerate}
              \item Allocate and initialize the TCB for the main thread.
              \item Allocate \textbf{both} the user-space and kernel-space stack space for the thread.
                    (part 2 in Figure \ref{fig:mos-process-memory-layout})
              \item Add the thread to the thread table.
              \item Set thread state to \texttt{CREATED}.
          \end{enumerate}
    \item (optionally) schedule to the new thread.
\end{enumerate}

A process has to be created from a `program' (an executable file) in order to be executed.
In MOS, the format called `ELF' is used as the executable format.

Thus, to be honest, the journey of a process should start from the function \texttt{elf\_create\_process}
located in \texttt{kernel/elf/elf.c} (corresponding to the first step above), but the code for
parsing an ELF file is a bit overcomplicated and not very interesting.

Because of that, we'll jump ahead and begin at the \texttt{process\_new} function.

\subsection{\texttt{process\_new}}

This function is defined in file \texttt{kernel/tasks/process.c}. It accepts several arguments,
the important ones among them are:

\begin{itemize}
    \item \texttt{process\_t *parent}: The parent of the process, this is a pointer to the PCB of the parent.
    \item \texttt{const char *name}: The name of the process, this is just a string that is used for debugging.
    \item \texttt{thread\_entry\_t entry}: The entry point of the process, this is the address of the
          \texttt{main} function of the program.
\end{itemize}

The function first allocates a PCB by calling \texttt{process\_allocate}. In that function, a
\texttt{process\_t} structure is allocated and initialized. The PID, magic number, name and
parent are set, then it calls \texttt{mm\_create\_user\_pgd} to create a user-mode page table,
which will become the `address space' of the process.

Going back to \texttt{process\_new}, several (precisely, 3) calls to \texttt{process\_attach\_ref\_fd}
are made to attach the standard input, output and error streams to the process.

Then the function calls \texttt{thread\_new} to create a main thread for the process. As you can see,
the \texttt{entry} is passed as the argument.

\begin{note}
    \item The function also passes \texttt{NULL} as the \texttt{arg} argument, which is an implementation
    detail that MOS handles arguments for the main thread differently.
\end{note}

After the main thread is created, the function allocates a heap by calling \texttt{mm\_alloc\_pages}
and then completes the initialization of the PCB by adding the PCB into the process table.

We'll now look at the \texttt{thread\_new} function.

\subsection{\texttt{thread\_new}}

This function is defined in file \texttt{kernel/tasks/thread.c}.

Similarly, it firstly calls \texttt{thread\_allocate} to allocate a TCB for the thread, which
initializes the \texttt{tid}, \texttt{magic}, \texttt{owner}, \texttt{state} and \texttt{mode}
fields.

After a TCB is allocated, the function allocates two types of stacks for the thread:

\begin{itemize}
    \item \textbf{Kernel Mode Stack}

          This is the stack used when the thread is running in kernel mode. A thread is running
          in `kernel mode' when it is executing a system call, or when a hardware interrupt, such as
          a timer interrupt or a keyboard interrupt, occurs.

    \item \textbf{User Mode Stack}

          This is the stack used when the thread is running in user mode, nothing special here?
\end{itemize}

\begin{tip}
    \item Pay attention to the different \texttt{mm\_} functions used to allocate the stacks.
    The kernel stack is allocated by \texttt{mm\_alloc\_pages}, while the user stack is
    allocated by \texttt{mm\_alloc\_zeroed\_pages}.
    \item The reason for this will be explained in the later chapters.
\end{tip}

After stacks are allocated, the function calls \texttt{platform\_context\_setup}, which is
a platform-specific function that setup the initial context of the thread.

Although we won't go into details into architecture-specific code, it's worth noting it's here
that the \texttt{entry} and \texttt{arg} arguments are used. For example, the entry will be
the initial instruction pointer of the thread and the \texttt{arg} will be pushed onto the stack
or passed in a register (depending on the architecture).

After the context is setup, the function adds the thread into the thread table, completing
the initialization of that thread.

\subsection{Thread States}

In MOS, a thread can be in one of the following states:

\begin{itemize}
    \item \texttt{CREATED}: thread is created or forked, but not ever started
    \item \texttt{READY}: thread can be scheduled
    \item \texttt{RUNNING}: thread is currently running
    \item \texttt{BLOCKED}: thread is blocked by a wait condition
    \item \texttt{DEAD}: thread is dead, and will be cleaned up soon by the scheduler
\end{itemize}

The state transition diagram of a thread is shown in Figure \ref{fig:thread-state-transition}
\footnote{Process termination with multiple running threads is not shown in this diagram}.

\begin{figure}
    \centering
    \begin{tikzpicture}[auto]

        \node[state] (C) at (0,-2) {CREATED};
        \node[state] (D) at (0, 2) {DEAD};

        \node[state] (U) at (3, 1) {RUNNING};
        \node[state] (R) at (9, 1) {READY};
        \node[state] (B) at (6,-3) {BLOCKED};

        \draw[->, thick] (-2, -2) -- (C);

        \path[->] (C) edge node {initial run} (U);
        \path[->] (U) edge node {exit} (D);

        \path[->] (U) edge [bend right] node {interrupted} (R);
        \path[->] (R) edge [bend right] node {scheduled} (U);
        \path[->] (U) edge [bend right] node {wait} (B);
        \path[->] (B) edge [bend right] node {wakeup} (R);
    \end{tikzpicture}
    \caption{Thread State Transition Diagram in MOS}
    \label{fig:thread-state-transition}
\end{figure}

\begin{note}
    \item MOS transitions a thread from \texttt{CREATED} \textbf{directly} to
    \texttt{RUNNING} without going through the \texttt{READY} state. This is different
    from what's in the lecture, where a thread is then \texttt{admitted}.

    This is because the scheduler in MOS require state as an indicator of whether a thread
    needs special setup before really jumping to the thread entry point.
\end{note}

In later chapters, we'll see what really happens when a thread is scheduled, blocked
or woken up. For now, we'll just focus on the process and thread creation.

\subsection{Process Address Space}

In MOS, a process has a virtual address space that is shared by all its threads. This is
similar to what Linux does. The underlying mechanism is called `paging' which we'll
also cover in later chapters.

The figure \ref{fig:mos-process-memory-layout} shows the memory layout of a process in MOS
after its creation.

\begin{figure}
    \definecolor{lightcyan}{rgb}{0.8,1,1}
    \definecolor{lightgreen}{rgb}{0.56,0.93,0.56}
    \definecolor{lightlightgreen}{rgb}{0.8,1,0.8}
    \definecolor{gray}{rgb}{0.7,0.7,0.7}
    \definecolor{lightred}{rgb}{1,0.7,0.71}

    % \memsection{end address}{start address}{height in lines}{text in box}{color}
    \newcommand{\memsection}[6][lrtb]{%
        \bytefieldsetup{bitheight=#4\baselineskip}%
        \bitbox[]{10}{
            \texttt{#2} \\ % end address
            \vspace{#4\baselineskip}
            \vspace{-2\baselineskip}
            \vspace{-#4pt}
            \texttt{#3} % start address
        }
        \bitbox[#1]{16}[bgcolor=#5]{\small #6}
    }

    \newcommand{\memgap}[2][lrtb]{
        \bytefieldsetup{bitheight=#2\baselineskip}
        \bitbox[#1]{16}[bgcolor=gray]{\small Gap}
    }

    Color Legend:
    \colorbox{gray}{\textbf{Unavailable}}
    \colorbox{lightred}{\textbf{Kernel Only}}
    \colorbox{lightgreen}{\textbf{User Read-Only}}
    \colorbox{cyan}{\textbf{User Read-Write}}

    \begin{center}
        \begin{bytefield}{24}
            \memsection{0xffffffff}{0xC0000000}{6}{lightred}{Kernel}\\
            \memgap{6}\\
            \begin{leftwordgroup}{2. Address determined\\ by the MOS Kernel}
                \begin{rightwordgroup}{4. Per-thread\\Memory Regions}
                    \memsection{\dots}{}{2}{lightred}{\scriptsize{Kernel-Mode Thread Stacks \dots}}\\
                    \memsection{}{}{2}{cyan}{\scriptsize{User-Mode Thread Stacks \dots}}
                \end{rightwordgroup}\\
                \begin{rightwordgroup}{4. \textbf{Main} Thread\\Memory Regions}
                    \memsection{}{}{2}{lightred}{\scriptsize{Kernel-Mode \textbf{Main} Thread Stack}}\\
                    \memsection{0x60020000}{\dots}{2}{cyan}{\scriptsize{User-Mode \textbf{Main} Thread Stack}}
                \end{rightwordgroup}\\
                \memgap{2}\\
                \begin{rightwordgroup}{3. Per-process\\Memory Regions}
                    \memsection[ltr]{}{}{5}{lightcyan}{\textit{Future} Heap \textit{Area}}\\
                    \memsection[lbr]{}{0x40000000}{3}{cyan}{\texttt{\large{$\uparrow$}} \\ Heap}
                \end{rightwordgroup}
            \end{leftwordgroup}\\
            \memgap{4}\\
            \begin{leftwordgroup}{1. Address statically\\defined by compiler}
                \memsection{\dots}{}{3}{cyan}{\texttt{.bss} Section}\\
                \memsection{}{}{3}{cyan}{\texttt{.data} Section}\\
                \memsection{}{}{2}{lightgreen}{\texttt{.rodata} Section}\\
                \memsection{}{0x08048000}{2}{lightgreen}{\texttt{.text} Section}
            \end{leftwordgroup}\\
            \memsection{}{0x00000000}{5}{gray}{Unavailable}\\
        \end{bytefield}
    \end{center}
    \caption{The memory layout of a process in MOS (Not to scale)}
    \label{fig:mos-process-memory-layout}
\end{figure}

\subsection{Exercises for Part 1}

\begin{exercise}
    \item Read the code in \texttt{process.c}, find the function \texttt{process\_dump\_mmaps()}
    \item Print the memory map of the \texttt{init} process in \texttt{mos\_start\_kernel}.
    \item Find out where is those memory map information is added to the process, and try
    print out a log message each time a new memory map is added.
\end{exercise}

\section{The famous \texttt{fork()} system call}

You may have known that the \texttt{fork()} system call is used to create a new process, which
seems to be the same as what we have just discussed. However, think about the following
question:

\begin{quote}
    What if we want to create a new process that is exactly the same as the current process,
    instead of creating a new process from a program?
\end{quote}

\dots TODO
