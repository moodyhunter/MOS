\chapter{Lab 1 - Setting Up the Development Environment}

In this chapter, we'll set up the development environment for the rest of our labs. Firstly
I'll introduce you to the tools we'll be using, and then, in the second part, we'll actually
install them.

\section{Introduction to the Tools}

MOS is an operating system, thus, only to prepare for a development environment is already
not an easy task (bruh). Several efforts have been made to make the process easier.

We are currently targeting the 32-bit \texttt{x86} architecture, the tools in table \ref{tab:tools}
are the ones we'll be using.

\begin{table}[h!]
    \centering
    \begin{tabular}{|l|c|c|}
        \hline \textbf{Tool}               & \textbf{Description}     & \textbf{Installation}            \\
        \hline \texttt{CMake}              & \ref{sec:cmake}          & \ref{sec:cmake-install}          \\
        \hline \texttt{i686-elf} toolchain & \ref{sec:cross-compiler} & \ref{sec:cross-compiler-install} \\
        \hline \texttt{NASM}               & \ref{sec:nasm}           & \ref{sec:nasm-install}           \\
        \hline \texttt{cpio}               & \ref{sec:cpio}           & \ref{sec:cpio-install}           \\
        \hline \texttt{qemu-system-i386}   & \ref{sec:qemu}           & \ref{sec:qemu-install}           \\
        \hline
    \end{tabular}
    \caption{Tools used in this lab}
    \label{tab:tools}
\end{table}

\subsection{CMake} \label{sec:cmake}

\begin{quote}
    CMake is an open-source, cross-platform family of tools designed to build, test and package
    software\footnote{https://cmake.org/}
\end{quote}

MOS uses CMake as the build system generator, it supports many build systems like `Make', `Ninja',
`Visual Studio' and `Xcode'.

\begin{note}
    \item It's the actual build system (e.g. `Make') that starts the compiler, linker, etc.,
    CMake is only to \textbf{generate} the configuration files for such build system.
\end{note}

We'll use `Make' as the build system for MOS in this tutorial, but
\textbf{you can use `Ninja' if you want}.

\subsection{NASM} \label{sec:nasm}

NASM is an assembler for x86 architecture. There are several files under `arch/x86`
that are written in NASM. It has a cleaner syntax than the GNU assembler (i.e. \texttt{as}).

\subsection{cpio} \label{sec:cpio}

Cpio is the format of an archive, and also the tool to create such archives. MOS uses cpio as
the initial root filesystem.

\subsection{qemu-system-i386} \label{sec:qemu}

QEMU is an open-source emulator, it also provides a gdb stub for debugging. It can be
installed via your Linux's package manager.

(e.g. \texttt{apt install qemu-system-i386} or \texttt{pacman -S qemu-system-x86}, \dots).

See its \href{https://www.qemu.org/download}{download page} for more details.

\subsection{i686-elf Toolchain} \label{sec:cross-compiler}

As its name suggests, this is a cross toolchain for `i686-elf' target. `i686' means the 32-bit
x86 architecture, and `elf' is the executable format. They together form the `target-triple' of
the toolchain.

\begin{tip}
    \item Most 64-bit Linux OSes have `target triple' of \texttt{x86\_64-pc-linux-gnu}, or
    \texttt{x86\_64-linux-gnu}.
    \item Read more about `target triple' at \url{https://wiki.osdev.org/Target_Triplet}.
\end{tip}

Unlike other applications (e.g. \texttt{bash} or \texttt{vim}) that they run on an existing
operating system and a standard C library (say, \texttt{glibc} or \texttt{musl}). MOS itself is
the operating system, thus there's not an existing OS for it to run on, neither a standard libc
(i.e. no \texttt{printf}, no \texttt{malloc} etc.) for it to use, considering you're directly
interacting with the CPU and the hardware.

This is called `bare-metal' environment, or `freestanding' environment, a `bare-metal' compiler
toolchain is exactly for this situation.

\begin{warning}
    \item One should never use a hosted compiler (e.g. the \texttt{gcc} installed on the lab machine)
    to cross-compile for a bare-metal environment when they are targeting a different architecture,
    (\texttt{x86} and \texttt{x86\_64} are different), it \textit{\textbf{may sometimes}} work, but
    you'll have to add a lot of unnecessary flags.

    \item See \url{https://wiki.osdev.org/Why_do_I_need_a_Cross_Compiler%3F} for more information.
\end{warning}

\section{Installating the Tools}

\subsection{CMake} \label{sec:cmake-install}

CMake should come with your Linux distribution's package manager, MOS requires at least version
3.20, but any newer version is recommended.

\subsection{NASM} \label{sec:nasm-install}

NASM can be installed via your Linux's package manager, the minimum version of NASM tested is \texttt{2.15.03}.
A pre-built binary from \href{https://www.nasm.us}{NASM's website} is also available.

\subsection{cpio} \label{sec:cpio-install}

cpio can also be installed via your Linux's package manager, the minimum version of cpio tested is \texttt{2.12}.

\subsection{qemu-system-i386} \label{sec:qemu-install}

TODO

\subsection{i686-elf Toolchain} \label{sec:cross-compiler-install}

Installing the cross-compiler toolchain is probably the most difficult part of this lab, but it's
once and for all, you don't have to do it again (unless you want to upgrade the toolchain).

There are majorly two ways to get this toolchain, either by building it from source or by downloading
a pre-built binary.

\subsubsection{Downloading a pre-built binary}

If you don't want to build the toolchain from source, you can download a pre-built
binary from \href{https://github.com/moodyhunter/i686-elf-prebuilt/releases}{moodyhunter/i686-elf-prebuilt} (choose the i686 one).

\begin{warning}
    \item Using pre-built binary saves time, but please consider doing so \textbf{only} if you trust the author.
    \item The above pre-built binary is built with GitHub Actions, and is built on Ubuntu 20.04.5 LTS (Image \texttt{ubuntu-20.04} version \texttt{20221027.1}).
\end{warning}

\subsubsection{Building From Source}

The source code of binutils and gcc can be found at \href{https://www.gnu.org/software/binutils}{GNU Binutils's Website}
and \href{https://gcc.gnu.org}{GNU GCC's Website} respectively.

\begin{note}
    \item For Arch Linux users, checkout
    \href{https://github.com/moodyhunter/repo/blob/main/moody/i686-elf-binutils/PKGBUILD}{i686-elf-binutils},
    \href{https://github.com/moodyhunter/repo/blob/main/moody/i686-elf-gcc/PKGBUILD}{i686-elf-gcc} and
    \href{https://github.com/moodyhunter/repo/blob/main/moody/i686-elf-gdb/PKGBUILD}{i686-elf-gdb}.
\end{note}

The script located at \texttt{docs/assets/i686-elf-toolchain.sh} downloads, compiles and installs them into the
directory specified by the \texttt{PREFIX} variable.
